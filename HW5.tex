%%%%%%%%%%%%%%%%%%%%%%%%%%%%%%%%%%%%%%%%%
% Short Sectioned Assignment
% LaTeX Template
% Version 1.0 (5/5/12)
%
% This template has been downloaded from:
% http://www.LaTeXTemplates.com
%
% Original author:
% Frits Wenneker (http://www.howtotex.com)
%
% License:
% CC BY-NC-SA 3.0 (http://creativecommons.org/licenses/by-nc-sa/3.0/)
%
%%%%%%%%%%%%%%%%%%%%%%%%%%%%%%%%%%%%%%%%%

%----------------------------------------------------------------------------------------
%	PACKAGES AND OTHER DOCUMENT CONFIGURATIONS
%----------------------------------------------------------------------------------------

\documentclass[paper=a4, fontsize=11pt]{scrartcl} % A4 paper and 11pt font size

\usepackage[T1]{fontenc} % Use 8-bit encoding that has 256 glyphs
\usepackage{fourier} % Use the Adobe Utopia font for the document - comment this line to return to the LaTeX default
\usepackage[english]{babel} % English language/hyphenation
\usepackage{amsmath,amsfonts,amsthm} % Math packages

\usepackage{sectsty} % Allows customizing section commands
\usepackage[top=5em]{geometry}
\allsectionsfont{\centering \normalfont\scshape} % Make all sections centered, the default font and small caps

\usepackage{fancyhdr} % Custom headers and footers
\pagestyle{fancyplain} % Makes all pages in the document conform to the custom headers and footers
\fancyhead{} % No page header - if you want one, create it in the same way as the footers below
\fancyfoot[L]{} % Empty left footer
\fancyfoot[C]{} % Empty center footer
\fancyfoot[R]{\thepage} % Page numbering for right footer
\renewcommand{\headrulewidth}{0pt} % Remove header underlines
\renewcommand{\footrulewidth}{0pt} % Remove footer underlines
\setlength{\headheight}{5pt} % Customize the height of the header

\numberwithin{equation}{section} % Number equations within sections (i.e. 1.1, 1.2, 2.1, 2.2 instead of 1, 2, 3, 4)
\numberwithin{figure}{section} % Number figures within sections (i.e. 1.1, 1.2, 2.1, 2.2 instead of 1, 2, 3, 4)
\numberwithin{table}{section} % Number tables within sections (i.e. 1.1, 1.2, 2.1, 2.2 instead of 1, 2, 3, 4)

\setlength\parindent{0pt} % Removes all indentation from paragraphs - comment this line for an assignment with lots of text

\usepackage{mathtools}
\usepackage{amssymb}
\usepackage{gensymb}
\usepackage{chngcntr}
\usepackage{csquotes}
\usepackage{flexisym}

\counterwithout{figure}{section}
%----------------------------------------------------------------------------------------
%	TITLE SECTION
%----------------------------------------------------------------------------------------

\newcommand{\horrule}[1]{\rule{\linewidth}{#1}} % Create horizontal rule command with 1 argument of height

\title{	
\normalfont \normalsize 
\textsc{Utah State University, Computer Science Department} \\ [25pt] % Your university, school and/or department name(s)
\horrule{0.5pt} \\[0.4cm] % Thin top horizontal rule
\huge CS 7910 Computational Complexity\\Assignment 5 \\ % The assignment title
\horrule{2pt} \\[0.5cm] % Thick bottom horizontal rule
}

\author{Gopal Menon} % Your name

\date{\normalsize\today} % Today's date or a custom date

\begin{document}

\maketitle % Print the title

\begin{enumerate}
\item (20 points) In this exercise, we consider the following \emph{Partition Problem}.\\
Given a set $A$ of $n$ numbers, the \emph{partition problem} is to decide whether $A$ can be partitioned into two subsets $A_1$ and $A_2$ (i.e., $A=A_1 \cup A_2$ and $A_1 \cap A_2 = \phi$) such that the sum of the numbers in $A_1$ is equal to the sum of the numbers in $A2$.\\
For example, suppose $A = \{5, 9, 3, 12, 11\}$; then we can partition $A$ into $A_1 = \{5, 3, 12\}$ and $A_2 = \{9, 11\}$.\\
Prove that the partition problem is NP-Complete (Hint: by a reduction from the subset-sum problem).\\

We know that the subset sum problem is NP-Complete. Let an instance of the subset sum problem be $(U, M)$, where $U$ is a set of numbers and we need to find a set of numbers $S$ such that $S \subseteq U$ and $\sum S = M$.

Let us construct an instance of the partition problem $(A)$, where $A$ is a set of numbers that needs to be partitioned into two sets such that the elements of each set add up to the same number. Construct the set $A$ in such a manner that it consists of all the elements in set $U$ with the addition of an element that has a value $\sum U - 2*M$. So the set $A$ will have one more element than the set $U$.

This construction can clearly be done in polynomial time. Also, given a certificate of the partition problem, which will be a subset of $A$, we can verify in $O(\left | A \right|)$, or polynomial time, that the certificate is correct. This shows that the partition problem is in the set NP. 

Consider the case where the instance of the subset sum problem is true. In the case of the partition problem that was constructed, we can create a subset $S\textprime$ such that $S\textprime = S \cup (\sum U - 2*M)$, where $\sum U - 2*M$ is the value of the additional element in set $A$, when compared with set $U$. The elements in set $S\textprime$ will sum up to $\sum S + \sum U - 2*M$ or $M + \sum U - 2*M$, since $\sum S = M$. The sum of the elements in $S\textprime$ will reduce to $\sum U - M$. The sum of all the elements in set $A$ will be $\sum U + (\sum U - 2*M)$ or $2 * \sum U - 2*M$. This can be simplified as $2*(\sum U - M)$. This means that the sum of the elements in set $A$ is twice the sum of the elements in set $S\textprime$. Or in other words, the set $A$ can be partitioned into two sets that sum up to the same number.

Consider the case where the set $A$ can be partitioned into two sets that sum up to the same number. We know that the set $A$ has been constructed by doing $U \cup (\sum U - 2* M)$. We do not know whether there exists a subset of $U$ that adds up to $M$. Each partition of set $A$ adds up to the same amount $\sum U - M$. Consider the case of the subset sum instance. The two partitions of the set $A$, will be mapped to two portions of the set $U$, which will sum up to $\sum U - M$ and $(\sum U - M) - (\sum U - 2* M)$. The sum of the second portion of the set $U$ may be simplified to $M$. This shows that if the partition problem instance is true, the subset sum instance is also true.

So we have shown that subset sum can be reduced to the partition problem in polynomial time. This means that the partition problem is in NP-Hard. Since we already know that the partition problem is in NP, it follows that the partition problem is in NP-Complete.

\end{enumerate}

%----------------------------------------------------------------------------------------

\end{document}